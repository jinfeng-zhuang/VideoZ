\documentclass{article}

\usepackage[UTF8]{ctex}
\usepackage[top=3cm,bottom=2cm,left=2cm,right=2cm]{geometry}

\title{傅立叶变换}
\author{庄金峰}

\begin{document}

\maketitle

\section{启发}

一个矩形波,可以由很多个正弦波叠加慢慢逼近,在WIKI中有展示。一个正弦波,对应一个圆周运动在X轴上的投影。正弦波的周期对应角速度,振幅对应半径,相位对应起始位置。所以用一个圆表示一个正弦波是可以的。正弦波按时间变化和圆周的角速度建立了时域到频域的转化通道。

\section{正交坐标系}

参照空间直角坐标系,如果两个向量的点乘等于0,说明二者相互垂直。

常量,正弦函数,余弦函数在周期范围内相乘,叠加之后的值等于0,说明这三者相互垂直,可以构成坐标系。

\subsection{定义}

综合上面描述,f(x) 可以分解为一个常量,多个正弦和余弦的叠加:

$f(x) = a_{0} + a_{1}cosx + b_{1}sinx + a_{2}cos2x + b_{2}sin2x + ... + a_{n}cosx + b_{n}sinx$

$f(x) = sin(x)$

\subsection{计算系数}

计算 $a_{0}$:

$\int_{-\pi}^{\pi} f(x)dx = \int_{-\pi}^{\pi} (a_{0} + a_{1}cosx + b_{1}sinx + a_{2}cos2x + b_{2}sin2x + ... + a_{n}cosx + b_{n}sinx) dx$

$\int_{-\pi}^{\pi} f(x)dx = \int_{-\pi}^{\pi} a_{0} dx$

$a_{0} = \frac{1}{2\pi} \int_{-\pi}^{\pi} f(x) dx$

计算 $a_{n}, b_{n}$:

$\int_{-\pi}^{\pi} cos nx f(x)dx = \int_{-\pi}^{\pi} (a_{0} + a_{1}cosx + b_{1}sinx + a_{2}cos2x + b_{2}sin2x + ... + a_{n}cosx + b_{n}sinx) cosnx dx$

$\int_{-\pi}^{\pi} cos nx f(x)dx = \int_{-\pi}^{\pi} a_{n}cos^2nxdx$

$a_{n} = \frac{1}{\pi} \int_{-\pi}^{\pi} cosnx f(x) dx$

\section{离散余弦变换}

相关性越强,DCT应用后,右下角零值越多,使用“之”字型扫描,可以得到很多连续的零,这里就可以减少数据量。

\subsection{Matlab实验}

Matlab是比较纯粹的学习和数据参考平台。

\subsection{C实验}

\end{document}

